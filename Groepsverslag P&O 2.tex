\documentclass[kulak]{kulakarticle} % options: kulak (default) or kul

\usepackage[dutch]{babel}

\title{Smart Fire Extinguisher}
\author{TEAM 6: Anna-Laura, Emile, Jérôme, Jesse}
\date{Academiejaar 2022 -- 2023}
\address{
	\textbf{Groep Wetenschap \& Technologie Kulak} \\
	Ingenieurswetenschappen \\
	P\&0 2}


\begin{document}
	
	\maketitle
	
	
	
	\section*{Inleiding}
	\section{Probleem Schetsen}
	
	De klant verwacht een apparaat dat zelfstandig branden kan detecteren en die kan blussen. Hiervoor moet het de exacte locatie van de brand kunnen vaststellen en de arm in de juiste richting richten (horizontale rotatie). Waarna het de tweede arm beweegt om de hoek zodanig te krijgen dat het water op de exacte locatie van de brand terecht komt (verticale rotatie). Het apparaat moet water vanuit een jerrycan in de richting van de brand spuiten en zelf stoppen wanneer de brand geblust is. 
	
	Alles moet automatisch werken, maar er moet ook een manuele override zijn waarbij het apparaat volledig manueel kan worden bestuurd en worden uitgeschakeld. Al dit moet gebeuren in communicatie met een PC. 
	
	
	\subsection{Huidige problemen bij Sprinklers}
	
	Sprinklers zijn heel handig wegens hun grote bestrijkingsgebied, waarbij ze in hele korte tijd water op een groot gebied kunnen doen neerdalen. Hier komen wel verschillende nadelen bij, zoals de hoogoplopende kosten van het aanleggen van het sprinklersysteem. Ook het onderhouden ervan is allesbehlave goedkoop. 
	
	Bovendien kan er veel waterschade zijn na een brand, wat zeker niet altijd nodig is. Voor een klein brandje zal een veel groter gebied met water besproeid worden, wat voor meer schade zorgt dan nodig is.
	
	
	\subsection{Blusplatform}
	
	Om deze nadelen te omzeilen opteren we voor een automatisch blusplatform. Het detecteerd branden en besproeit die gericht net water, zonder veel waterschade. Ook is de prijs ervan veel goedkoper, omdat er maar een leiding moet lopen naar de brandblusser.
	
	
	\subsection{Voordelen van het Blusplatform}
	
	tekst
	
	
	
	\section{Ontwerp en Materialen}
	
	productspecificaties hier schrijven wss
	
	
	\subsection{Ontwerpproces}
	
	tekst
	
	
	\subsection{Materiaalselectie}
	
	tekst
	
	
	\subsection{Solid Edge}
	
	alle onderdelen apart? of enkel volledig ontwerp?
	
	
	
	\section{Elektrisch Circuit}
	
	tekst
	
	
	\subsection{Motoren}
	
	tekst
	
	
	\subsection{Sensoren en Webcams}
	
	tekst
	
	
	\subsection{Moederbord}
	
	tekst
	
	
	\subsection{Bekabeling}
	
	tekst
	
	
	
	\section{Programmeercode}
	
	tekst
	
	
	\subsection{LabView}
	
	tekst
	
	
	\subsection{Python}
	
	tekst
	
	
	\subsection{Raspberry Pi}
	
	tekst
	
	
	
	\section{Resultaten}
	
	tekst
	
	
	\subsection{Prototype}
	
	tekst
	
	
	\subsection{Resultaten Demo}
	
	tekst
	
	
	
	\section{Financieel Rapport}
	
	tekst
	
	
	
	\section{Mogelijke Verbeteringen}
	
	tekst
	
	
	
	\section*{Besluit}
	
	Afsluitende tekst.
	
	
	
	\bibliography{}
	
	
	
	
\end{document}