\documentclass[kulak]{kulakarticle} % options: kulak (default) or kul

\usepackage[dutch]{babel}
\usepackage{amsmath}

\title{Smart Fire Extinguisher}
\author{TEAM 6: Anna-Laura, Emile, Jérôme, Jesse}
\date{Academiejaar 2022 -- 2023}
\address{
	\textbf{Groep Wetenschap \& Technologie Kulak} \\
	Ingenieurswetenschappen \\
	P\&0 2}



\begin{document}
\maketitle



\section*{Inleiding}        
Bij de brandbestrijding in grote warenhuizen worden momenteel \textbf{sprinklers} gebruikt. Deze zijn vastgemaakt aan het plafond van het gebouw en zijn aan de waterleiding aangesloten om bij brand water te doen neerdalen. Ze werken heel efficiënt, maar hebben wel enkele grote nadelen. De voornaamste zijn de kost van de aanleg en het onderhoud van alle leidingen die de sprinklers van water voorzien. Ook de verhoogde kans op waterschade bij het springen van een van de vele waterleidingen maakt sprinklers ietwat minder aantrekkelijk. Bovendien is het bestrijkingsgebied bij kleine branden niet proportioneel met de grootte van de brand waardoor er ook meer waterschade kan optreden dan nodig is. \\

Daarom zijn wij op zoek gegaan naar een efficiëntere manier om branden te blussen. Een \textbf{Smart Fire Extinguisher}, die zelf de branden kan detecteren, lokaliseren en gericht kan blussen. Zo zou 1 apparaat (met dus maar 1 aansluiting op de waterleiding of een eigen waterreservoir) een groot oppervlakte brandveilig kunnen maken. Dit zou het veel goedkoper maken voor de eigenaar die geen eindeloos lange waterleidingen moet aanleggen en onderhouden. De totale kost voor grote warenhuizen zou dus veel lager liggen en de kans op waterschade bij het springen van waterleidingen is veel kleiner.


\section{Ontwerp en Materialen}
\subsection{Ontwerpproces}
% deeltje uit tussentijds verslag halen + toevoegen wrm gekozen voor bepaalde lengte van arm want we kennen nu de hoogte van de te vullen cilinders 0,5m), en we kiezen ervoor om ons apparaat een halve meter te maken dus: berekening toevoegen van hoe ver we kunnen geraken als we geen arm gebruiken en het water op een magische wijze op 45 graden wordt weggesporten en dan kijken of dit ver genoeg is en de lengte (en eig ook hoogte) daarvan laten afhangen. 
%ook berekening toevoegen waarom pomp op hoogst mogelijke positie te plaatsen
%klein deeltje schrijven over waarom camera laten meedraaien
%berekeningen voor diameter van spuit toevoegen 

\subsection{Materiaalselectie}
%nog toevoegen wat nieuw gekocht + bij alles uitleggen waarom gekocht
Arduino Nano 33 iot \\
Breadboard Full-size \\
Membraanpomp 12V 4.8 bar \\
2x Micro Metal Gear Motor 100:1 HP \\
Jerrycan 10L\\
Whadda WPSE470 waterflowsensor \\
USB Webcam 1080P \\
Powerbank\\
Step-Down Voltage Regulator \\
Flexibele slang 10mm \\
Slangenklemmen 10mm \\
MDF \\
Arms with 33 3 mm holes spaced 5 mm apart (10x170 mm) \\
Four-hole L-shaped brackets \\
MakerBeam profiel 200 mm \\
MakerBeam profiel 300 mm \\
MakerBeam Hoekverbinding 90° \\
MakerBeam Hoekverbinding 90° buitenhoek \\


\subsection{Ontwerp in Solid Edge}
%paar belangrijke onderdelen (eig vooral spuitstuk en mss arm ofzo) apart invoegen + volledige ontwerp insteken


\section{Elektrisch Circuit}
%schematische voorstelling van het elektrisch circuit toevoegen (er bestaan wel websites voor ofzo)


\section{Python}
%alle code die in python is geschreven uitleggen + de berekeningen waarop het is gebaseerd


\section{Arduino}
%alle code in de arduino uitleggen + berekeningen waarop gebaseerd is


\section{LabView}
%gwn code wa uitleggen?



\section{Prototype}
%zeggen hoe ons prototype werkte en niet werkte


\subsection{Aanpassingen}
%wat hebben we dan aangepast aan ons prototype


\section{Resultaten Demo}





\section{Financieel Rapport}





\section{Mogelijke Verbeteringen}




\bibliography{}




\end{document}